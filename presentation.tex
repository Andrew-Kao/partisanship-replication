\documentclass{beamer}

% *****************************************************************
% Estout related things
% *****************************************************************
\newcommand{\sym}[1]{\rlap{#1}}% Thanks to David Carlisle

\let\estinput=\input% define a new input command so that we can still flatten the document

\newcommand{\estwide}[3]{
		\vspace{.75ex}{
			\begin{tabular*}
			{\textwidth}{@{\hskip\tabcolsep\extracolsep\fill}l*{#2}{#3}}
			\toprule
			\estinput{#1}
			\bottomrule
			\addlinespace[.75ex]
			\end{tabular*}
			}
		}	

\newcommand{\estauto}[3]{
		\vspace{.75ex}{
			\begin{tabular}{l*{#2}{#3}}
			\toprule
			\estinput{#1}
			\bottomrule
			\addlinespace[.75ex]
			\end{tabular}
			}
		}

% Allow line breaks with \\ in specialcells
	\newcommand{\specialcell}[2][c]{%
	\begin{tabular}[#1]{@{}c@{}}#2\end{tabular}}

% *****************************************************************
% Custom subcaptions
% *****************************************************************
% Note/Source/Text after Tables
\usepackage{verbatim}
\newcommand{\figtext}[1]{
	\vspace{-1.9ex}
	\captionsetup{justification=justified,font=footnotesize}
	\caption*{\hspace{6pt}\hangindent=1.5em #1}
	}
\newcommand{\fignote}[1]{\figtext{\emph{Note:~}~#1}}

\newcommand{\figsource}[1]{\figtext{\emph{Source:~}~#1}}

% Add significance note with \starnote
\newcommand{\starnote}{\figtext{* p < 0.1, ** p < 0.05, *** p < 0.01. Standard errors in parentheses.}}



\usepackage{amsmath,amsthm,amssymb} %Misc math symbols
\usepackage{mathtools}
\usepackage[utf8]{inputenc}
\usepackage{caption} 				%Inserting multiple figures
\usepackage{subcaption}
\usepackage{listings}
\usepackage{tikz, pgfplots} 			%Drawing pictures
%\usepackage{fancyhdr} 				%Header style
%\usepackage{enumitem}				%Enumerating using letters
%\usepackage{mathrsfs}				%Fonts
%\usepackage{listings}
\usepackage{booktabs}
\usepackage{float}
\usepackage{import}
\usepackage{graphicx}
\usepackage{rotating}
\usepackage{pdflscape}
\usepackage{hyperref}
\usepackage{graphicx}
\usepackage{epstopdf}
\usepackage{adjustbox} 
\usepackage{amsmath} 
\usepackage{caption}
\usepackage{subcaption}
\usepackage{adjustbox}


 
\usepackage[utf8]{inputenc}
 
 
%Information to be included in the title page:
\title{Replicating Leigh 2008 - "Estimating the impact of gubernatorial partisanship on policy settings
and economic outcomes: A regression discontinuity approach" }
\subtitle{}
\author{Andrew Kao}
\institute{University of Chicago}
\date{Winter 2019}
 
 
 
\begin{document}
 
\frame{\titlepage}
 
\begin{frame}
\frametitle{Motivation and Identification}

\begin{itemize}
\item How ideological are governors in their actions?
	\begin{itemize}
	\item Downs model 
	\item Roemer model
	\end{itemize} 
\item Observational panel data
	\begin{itemize}
	\item Data from 1941 to 2004
	\item 30 outcomes for all 50 states
	\end{itemize}
\item Regression discontinuity
\end{itemize}
\end{frame}

\begin{frame}
\frametitle{Descriptives}
   
\scriptsize
    \renewcommand\arraystretch{0.75}
    \setlength\tabcolsep{2pt}
%\resizebox{\linewidth}{!}{
%\begin{table}[!htbp]
%\caption{Post Claim Earnings }
\include{output/tables/summary}
%\label{table18}
%{\textbf{Note:}  Observations only include those with non-missing data for all variables. Sample is constrained to control and claimant incentive arms. (* p $<$ .10, ** p $<$ .05, *** p $<$ .01) } 
%\end{table}
%}


\end{frame}

\begin{frame}
\frametitle{Regressions}

\scriptsize
    \renewcommand\arraystretch{0.65}
    \setlength\tabcolsep{1pt}
    \tiny
\begin{center}
\begin{tabular}{lccccc}
\hline \noalign{\smallskip}Dep Var & (1) & (2) & (3) & (4) & (5)\\
\noalign{\smallskip}\hline \noalign{\smallskip}Unionization Rate & \begin{scriptsize}-0.3040\end{scriptsize} & \begin{scriptsize}-0.2868\end{scriptsize} & \begin{scriptsize}-0.3248\end{scriptsize} & \begin{scriptsize}-0.3409\end{scriptsize} & \begin{scriptsize}-0.3124\end{scriptsize}\\
 & \begin{scriptsize}(0.2427)\end{scriptsize} & \begin{scriptsize}(0.2467)\end{scriptsize} & \begin{scriptsize}(0.2449)\end{scriptsize} & \begin{scriptsize}(0.2473)\end{scriptsize} & \begin{scriptsize}(0.3007)\end{scriptsize}\\
\noalign{\smallskip}Incarceration Rate & \begin{scriptsize}-9.5687\end{scriptsize} & \begin{scriptsize}-11.3970*\end{scriptsize} & \begin{scriptsize}-13.3071**\end{scriptsize} & \begin{scriptsize}-12.8029**\end{scriptsize} & \begin{scriptsize}-12.7568*\end{scriptsize}\\
 & \begin{scriptsize}(6.9422)\end{scriptsize} & \begin{scriptsize}(6.3649)\end{scriptsize} & \begin{scriptsize}(6.2159)\end{scriptsize} & \begin{scriptsize}(5.9304)\end{scriptsize} & \begin{scriptsize}(7.5291)\end{scriptsize}\\
\noalign{\smallskip}Executions per 100,000 & \begin{scriptsize}0.0040\end{scriptsize} & \begin{scriptsize}-0.0011\end{scriptsize} & \begin{scriptsize}-0.0009\end{scriptsize} & \begin{scriptsize}0.0044\end{scriptsize} & \begin{scriptsize}0.0006\end{scriptsize}\\
 & \begin{scriptsize}(0.0041)\end{scriptsize} & \begin{scriptsize}(0.0038)\end{scriptsize} & \begin{scriptsize}(0.0038)\end{scriptsize} & \begin{scriptsize}(0.0036)\end{scriptsize} & \begin{scriptsize}(0.0064)\end{scriptsize}\\
\noalign{\smallskip}Log State Transfers per Capita & \begin{scriptsize}-0.0195\end{scriptsize} & \begin{scriptsize}-0.0052\end{scriptsize} & \begin{scriptsize}-0.0071\end{scriptsize} & \begin{scriptsize}-0.0028\end{scriptsize} & \begin{scriptsize}0.0127\end{scriptsize}\\
 & \begin{scriptsize}(0.0283)\end{scriptsize} & \begin{scriptsize}(0.0268)\end{scriptsize} & \begin{scriptsize}(0.0274)\end{scriptsize} & \begin{scriptsize}(0.0276)\end{scriptsize} & \begin{scriptsize}(0.0302)\end{scriptsize}\\
\noalign{\smallskip}Log State UI Payments per Capita & \begin{scriptsize}-0.0032\end{scriptsize} & \begin{scriptsize}0.0074\end{scriptsize} & \begin{scriptsize}0.0082\end{scriptsize} & \begin{scriptsize}0.0069\end{scriptsize} & \begin{scriptsize}-0.0168\end{scriptsize}\\
 & \begin{scriptsize}(0.0244)\end{scriptsize} & \begin{scriptsize}(0.0244)\end{scriptsize} & \begin{scriptsize}(0.0242)\end{scriptsize} & \begin{scriptsize}(0.0244)\end{scriptsize} & \begin{scriptsize}(0.0310)\end{scriptsize}\\
\noalign{\smallskip}Percent Population on Welfare & \begin{scriptsize}0.1412\end{scriptsize} & \begin{scriptsize}0.1433\end{scriptsize} & \begin{scriptsize}0.1607*\end{scriptsize} & \begin{scriptsize}0.1618*\end{scriptsize} & \begin{scriptsize}0.0716\end{scriptsize}\\
 & \begin{scriptsize}(0.1003)\end{scriptsize} & \begin{scriptsize}(0.0900)\end{scriptsize} & \begin{scriptsize}(0.0927)\end{scriptsize} & \begin{scriptsize}(0.0963)\end{scriptsize} & \begin{scriptsize}(0.1104)\end{scriptsize}\\
\noalign{\smallskip}Log Real State Income Tax Receipt per Capita & \begin{scriptsize}-0.0665\end{scriptsize} & \begin{scriptsize}-0.0403\end{scriptsize} & \begin{scriptsize}-0.0305\end{scriptsize} & \begin{scriptsize}-0.0264\end{scriptsize} & \begin{scriptsize}-0.0992\end{scriptsize}\\
 & \begin{scriptsize}(0.0415)\end{scriptsize} & \begin{scriptsize}(0.0380)\end{scriptsize} & \begin{scriptsize}(0.0369)\end{scriptsize} & \begin{scriptsize}(0.0373)\end{scriptsize} & \begin{scriptsize}(0.0612)\end{scriptsize}\\
\noalign{\smallskip}Log Real Other Income Tax Receipt per Capita & \begin{scriptsize}0.0068\end{scriptsize} & \begin{scriptsize}0.0201\end{scriptsize} & \begin{scriptsize}0.0227\end{scriptsize} & \begin{scriptsize}0.0201\end{scriptsize} & \begin{scriptsize}0.0430\end{scriptsize}\\
 & \begin{scriptsize}(0.0295)\end{scriptsize} & \begin{scriptsize}(0.0279)\end{scriptsize} & \begin{scriptsize}(0.0278)\end{scriptsize} & \begin{scriptsize}(0.0278)\end{scriptsize} & \begin{scriptsize}(0.0394)\end{scriptsize}\\
\noalign{\smallskip}Log Real Non-Tax Revenue per Capita & \begin{scriptsize}0.0042\end{scriptsize} & \begin{scriptsize}0.0161\end{scriptsize} & \begin{scriptsize}0.0155\end{scriptsize} & \begin{scriptsize}0.0186\end{scriptsize} & \begin{scriptsize}0.0119\end{scriptsize}\\
 & \begin{scriptsize}(0.0359)\end{scriptsize} & \begin{scriptsize}(0.0346)\end{scriptsize} & \begin{scriptsize}(0.0358)\end{scriptsize} & \begin{scriptsize}(0.0338)\end{scriptsize} & \begin{scriptsize}(0.0329)\end{scriptsize}\\
\noalign{\smallskip}Log Real State Revenue per Capita & \begin{scriptsize}-0.0365\end{scriptsize} & \begin{scriptsize}-0.0235\end{scriptsize} & \begin{scriptsize}-0.0216\end{scriptsize} & \begin{scriptsize}-0.0192\end{scriptsize} & \begin{scriptsize}-0.0726*\end{scriptsize}\\
 & \begin{scriptsize}(0.0365)\end{scriptsize} & \begin{scriptsize}(0.0332)\end{scriptsize} & \begin{scriptsize}(0.0323)\end{scriptsize} & \begin{scriptsize}(0.0315)\end{scriptsize} & \begin{scriptsize}(0.0385)\end{scriptsize}\\
\noalign{\smallskip}\hline\end{tabular}\\
\end{center}

\textbf{Note:} Column 5 is the regression discontinuity (margin of victory $<$ 30\%). Columns 1 - 5 include state and year fixed effects \& cluster errors by state level. Columns 2 - 5 include demographic controls, columns 3 - 5 include controls for local legislature partisanship, columns 4 - 5 include controls for state representatives in Congress.

\end{frame}

\end{document}